%Copyright (c) 2017 Lizzie Stone
%
%Permission is hereby granted, free of charge, to any person obtaining a copy
%of this software and associated documentation files (the "Software"), to deal
%in the Software without restriction, including without limitation the rights
%to use, copy, modify, merge, publish, distribute, sublicense, and/or sell
%copies of the Software, and to permit persons to whom the Software is
%furnished to do so, subject to the following conditions:
%
%The above copyright notice and this permission notice shall be included in all
%copies or substantial portions of the Software.
%
%THE SOFTWARE IS PROVIDED "AS IS", WITHOUT WARRANTY OF ANY KIND, EXPRESS OR
%IMPLIED, INCLUDING BUT NOT LIMITED TO THE WARRANTIES OF MERCHANTABILITY,
%FITNESS FOR A PARTICULAR PURPOSE AND NONINFRINGEMENT. IN NO EVENT SHALL THE
%AUTHORS OR COPYRIGHT HOLDERS BE LIABLE FOR ANY CLAIM, DAMAGES OR OTHER
%LIABILITY, WHETHER IN AN ACTION OF CONTRACT, TORT OR OTHERWISE, ARISING FROM,
%OUT OF OR IN CONNECTION WITH THE SOFTWARE OR THE USE OR OTHER DEALINGS IN THE
%SOFTWARE.


\documentclass[titlepage 12pt]{article}
\textwidth = 450pt
\oddsidemargin = 5pt
\textheight = 650pt
\voffset = -40pt

%-------------------------Packages-----------------------------------------%

\usepackage{titling}
\usepackage{graphicx}
%\usepackage{natbib}
\usepackage{sectsty}
\setcounter{secnumdepth}{0}
\usepackage[labelfont=bf]{caption}
\usepackage{float}
\usepackage{amsmath}
\usepackage{minted}
\usepackage[
backend=biber,
style=ieee,
citestyle=numeric
]{biblatex}
 
%\addbibresource{project-specs.bib}
\addbibresource{project-specs.bib} 

%\setcitestyle{square}

%-------------------------Titles-------------------------------------------%

\sectionfont{\fontsize{12}{11}\selectfont\underline\sffamily}
\subsectionfont{\fontsize{12}{11}\selectfont\underline\normalfont\sffamily}
\subsubsectionfont{\fontsize{10}{11}\selectfont\underline\normalfont\sffamily}

\setlength{\droptitle}{10em} 
\title{Backpackable Robot Boat for Sonar Surveys\\ Outline Project Specification\\ CS39440}
\author{ Elizabeth Stone (eas12)\\FH56 Space Science and Robotics\\ \\Supervisor: Mark Neal (mjn) }
\date{Revision 1.0 - Final\\ \today\vspace{-3em} }

%------------------------Headers and footers-------------------------------%

\usepackage{fancyhdr}
 
\pagestyle{fancy}
\fancyhf{}
\rhead{{\fontfamily{cmss}\selectfont Elizabeth Stone (eas12)}}
\lhead{{\fontfamily{cmss}\selectfont Outline Project Specification}}
\rfoot{{\fontfamily{cmss}\selectfont Page \thepage}}
 
%-------------------------Title Page---------------------------------------%

\begin{document}
{\fontfamily{cmss}
\selectfont


\begin{titlepage}
	\clearpage\maketitle
\thispagestyle{empty}
	 \vspace{250pt}

%\section*{Abstract}
%	This report investigates the two and three body problem.
	
\end{titlepage}

%-----------------------Main Text------------------------------------------%

%\tableofcontents

%\tableofcontents

%\newpage

\section{Project Description}

The AqASS (Aquatic Area Scanning System) project will produce a control system for a small, lightweight, motor-powered robotic boat to enable the boat to autonomously scan any given body of water, with selectable parameters. 
\\ \noindent
\\Academics at Aberystwyth University Geography department studied Scottish lochs two and a half years ago using a  lightweight remote control boat. The study required the boat to scan the lochs using sonar sensors to build up an image of the bottom of the lochs and to locate logs on the loch floor (in order to investigate climate change using dendrochronology)\cite{rbates14}. In that study, the boat was controlled by RC. This project aims to make the boat fully autonomous in order to increase efficiency.
\\ \noindent
\\The most basic aim for this project is to develop code to make the boat travel in a straight line towards a selected location on a body of water. This can then be built upon to make the boat travel to a series of way-points.
\\ \noindent
\\Building on this, the project will then develop an algorithm for automatically selecting the most efficient course for boat to take to scan a selected area (as defined by a series of coordinates). 
\\ \noindent
\\It will then investigate telemetry and the design of a user interface that allows the user to communicate with the boat while it is on the water, and easily select the route the user wishes the robot to take. 
\\ \noindent
\\It may also be useful to investigate semi-autonomous functions that can be used on the boat before and after starting the area scanning, such as heading holding, station keeping or return home functions, to aid in deployment of the system.
      \\ \noindent    
\\Suitable development methodologies will need to be investigated early on in this project.

\section{Proposed Tasks}
The following tasks are proposed:
\begin{itemize}
\item \textbf{Investigate Development Methodologies}

\item \textbf{RC Lake Tests and environment research.} Investigate the environment encountered by the boats and the factors that affect their motion on the water (e.g. current, wind, waves), and what can be done to detect and account for them.

\item \textbf{Investigate OS and programming languages.} Research which languages are most suitable for the hardware given, and which OS would be best to use on the Pi (if any). 

\item \textbf{Control System Development} 
	\begin{itemize}
	\item \textbf{Investigate types of navigation systems.} Research types of system and see which type is most suitable for this project (vector field systems, bearing systems etc). 
	\item \textbf{Investigate Control System Architectures} See which design would be best for this type of project. It may be that a modular approach will be necessary, which would allow easy swapping of hardware.
	\item \textbf{Write code to get boat to travel in a straight line to a way-point/series of way-points.} This will involve accounting for factors discovered during rc testing.
	\item \textbf{Investigate ways of representing and specifying body of water to be scanned and develop an algorithm to determine the most efficient route to scan the given area.} Specifying the body may be as simple as manually entering the data points, or as advanced as having a GUI where the user may draw the outline of the water on a map.The algorithm will likely depend on the shape of the body of water, current direction and other factors. It should also consider parameters such as desired resolution of scan (which would depend upon the sonar sample rate, speed of boat, and spacing of consecutive traversals of the area).
	\item \textbf{Research and Develop Telemetry System} Develop easy to use interface to pass information to and from boat. Investigate if long distance communications would be useful and viable.
	\item \textbf{Investigate plausibility of (and implement) collision avoidance system.} A system to prevent the boat from travelling into shallow waters, or crashing into objects in its path (e.g. other boats, buoys). This may involve processing the data from the sonar scanner(s), thus research will need to be done into how to process this data. It may also be necessary to develop a modular system so that sonar device may be swapped easily for another similar device.
	\end{itemize}

\item \textbf{Test control system} The methodology used will define the frequency of testing, but tests on one or more large bodies of water would be ideal nearer the end of the project. On land tests will be carried out throughout the rest of the project.
\item \textbf{Project Meetings and Project Diary} Weekly project meetings will be held with the supervisor, and a project diary will be kept to keep track of all parts of the project. The diary will be written in markdown and backups will be kept in a git repository.
\item \textbf{Demonstrations}
\end{itemize}




\section{Deliverables}  
Deliverables expected from this project: 
\begin{itemize}
	\item Mid-project demonstration
	\item Control system software
	\item Usable user interface and telemetry software
	\item Collision avoidance software
	\item Final report
	\item Final demonstration
\end{itemize}


%------------------------Bibliography--------------------------------------%    
    
\newpage
\raggedright
\addcontentsline{toc}{section}{Annotated Bibliography}
\printbibliography[title={Annotated Bibliography}]
%\bibliographystyle{ieee}


}
\end{document}



